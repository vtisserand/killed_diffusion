In this study, we implement two numerical schemes for stochastic differential equations where the diffusion stops if it escapes from a defined domain. Theoretical results are given and sketches of the proofs are highlighted. We apply these results to a quantitative finance problem: the pricing of path-dependent option, in particular we play with an up-and-out call and a double no-touch call. The code is made public and accessible in a \href{https://github.com/vtisserand/killed_diffusion}{GitHub repository}.

\section{Killed diffusion, simulation schemes, what are we talking about?}

We are investigating a paper from Emmanuel Gobet published in 2000 \cite{gobet2000weak}. We give a short finance introduction on some exotic options before providing proofs and delving into numerical schemes to approximate such payoffs.

\subsection{Getting started -- why?}

We are interested in a path-dependent payoff on an asset with diffusion $(X_t)_{0\leq t \leq T}$.
In particular, we kill the diffusion when it leaves the open set $D$.\newline

With $\tau \coloneqq \inf\{t > 0 : X_t \notin D\}$, this leads to the computation of $\mathbb{E}\left[\mathbbm{1}_{T < \tau} f(X_T)\right].$ The direct application in finance is with barrier and digital options that pay respectively the underlying asset, one unit of a specified currency, conditional on the barrier levels not having been broken.


Barrier options are either up, down, and in, out: an up-and-out barrier option with barrier $B$ leads to the boundary condition $V(B, t) = 0 \quad \forall t \leq T.$ With $D = \left(-\infty, B\right]$,
the risk-neutral evaluation of the up-and-out call gives a price at time $t$: \[C_{up-out } = \mathbb{E}^{\mathbb{Q}}\left[e^{-\int_t^T r_s ds} (S_T-K)^{+} \mathbbm{1}_{T < \tau} \vert \mathcal{F}_t\right]\]

where $\tau = \inf\{t > 0 : S_t \notin D\} = \inf\{t > 0 : S_t > B\}$ if we make the assumption that $S_0 \leq B$.

Similarly, we can define an up-and-in call option using $D = \left[B, +\infty\right)$, usually with $S_0 \geq B$. Note that if $K \geq B$, the option simply becomes a vanilla call option of strike $K$ and maturity $T$.

For simplicity, we write $\mathbb{E}(.)$ as the expectation taken under risk-neutral measure $\mathbb{Q}$.\\

Overall, being able to study such payoffs opens the \textbf{realm of barrier options} as well as no-touch and one-touch options that are some path-dependent \textbf{binary options} -- the payoff is either $\$1M$ or $\$0$ based on a condition (\textit{e.g.} the EURUSD price remained under $1.087$ up until maturity of the option) being filled: these are high leverage speculative instruments that can lead to significant gains for those who master their use.\\

The paper implements two discretisation schemes to price such payoffs using Monte Carlo methods: a continuous Euler scheme and a discrete Euler scheme. We will be implementing those algorithms and running a set of experiments to evaluate the claims of the paper: do we actually observe a $o(N^{-1})$ convergence for the continuous scheme while only a $o(N^{-1/2})$ for the discrete scheme? How do the simulation prices compare to the theoretical closed-form formulas under Black-Scholes dynamics?

\subsection{Deriving the path-dependent options cost under Black-Scholes}

Let's assume a geometric brownian motion dynamic for the underlying asset: \[dS_t = r S_t dt + \sigma S_t dW_t\]

For all barrier options, an arbitrage argument gives the following parity formula between the prices of barrier options and vanilla call and put options:

\[
\begin{array}{l}
C_{\text {up-in }}(t)+C_{\text {up-out }}(t)=\mathrm{e}^{-(T-t) r} \mathbb{E}\left[\left(S_{T}-K\right)^{+} \mid \mathcal{F}_{t}\right] \\
C_{\text {down-in }}(t)+C_{\text {down-out }}(t)=\mathrm{e}^{-(T-t) r} \mathbb{E}\left[\left(S_{T}-K\right)^{+} \mid \mathcal{F}_{t}\right] \\
P_{\text {up-in }}(t)+P_{\text {up-out }}(t)=\mathrm{e}^{-(T-t) r} \mathbb{E}\left[\left(K-S_{T}\right)^{+} \mid \mathcal{F}_{t}\right] \\
P_{\text {down-in }}(t)+P_{\text {down-out }}(t)=\mathrm{e}^{-(T-t) r} \mathbb{E}\left[\left(K-S_{T}\right)^{+} \mid \mathcal{F}_{t}\right]
\end{array}
\]

where the price of the European call, resp. put option with strike price $K$ are obtained from the Black-Scholes formula.
Consequently, in what follows we will only compute the prices of the up-and-out barrier call.

Let's compute the price of the up-and-out barrier call option.

Introducing the maximum of the geometric brownian motion: \[M_{0}^{T}=\max _{0 \leq u \leq T} S_{u}\]

We have the following proposition:

\begin{proposition} When $K \leq B$, the price of the up-and-out barrier call option with maturity $T$, strike price $K$ and (upper) barrier level $B$ is given by:

\[
\begin{aligned}
& \mathrm{e}^{-(T-t) r} \mathbb{E}\left[\left(S_{T}-K\right)^{+} \mathbbm{1}_{\left\{M_{0}^{T}<B\right\}} \mid \mathcal{F}_{t}\right] = \\
& \mathbbm{1}_{\left\{M_{0}^{t}<B\right\}} \mathrm{Bl}\left(S_{t}, K, r, T-t, \sigma\right)-S_{t} \mathbbm{1}_{\left\{M_{0}^{t}<B\right\}} \Phi\left(\delta_{+}^{T-t}\left(\frac{S_{t}}{B}\right)\right) \\
& -B\left(\frac{B}{S_{t}}\right)^{2 r / \sigma^{2}} \mathbbm{1}_{\left\{M_{0}^{t}<B\right\}}\left(\Phi\left(\delta_{+}^{T-t}\left(\frac{B^{2}}{K S_{t}}\right)\right)-\Phi\left(\delta_{+}^{T-t}\left(\frac{B}{S_{t}}\right)\right)\right) \\
& +\mathrm{e}^{-(T-t) r} K \mathbbm{1}_{\left\{M_{0}^{t}<B\right\}} \Phi\left(\delta_{-}^{T-t}\left(\frac{S_{t}}{B}\right)\right) \\
& +\mathrm{e}^{-(T-t) r} K\left(\frac{S_{t}}{B}\right)^{1-2 r / \sigma^{2}} \mathbbm{1}_{\left\{M_{0}^{t}<B\right\}}\left(\Phi\left(\delta_{-}^{T-t}\left(\frac{B^{2}}{K S_{t}}\right)\right)-\Phi\left(\delta_{-}^{T-t}\left(\frac{B}{S_{t}}\right)\right)\right) .
\end{aligned}
\]

where

\[
\delta_{ \pm}^{\tau}(s)=\frac{1}{\sigma \sqrt{\tau}}\left(\log s+\left(r \pm \frac{\sigma^{2}}{2}\right) \tau\right), \quad s>0
\]

Note that the price of the up-and-out barrier call option vanishes when $B \leq K$.

Note also that as $B \to \infty$, the price of the up-and-out barrier call option converges to the price of the European call option:
 
\[
\begin{aligned}
\lim _{B \rightarrow \infty} \mathrm{e}^{-(T-t) r} \mathbb{E}\left[\left(S_{T}-K\right)^{+} \mathbbm{1}_{\left\{M_{0}^{T}<B\right\}} \mid \mathcal{F}_{t}\right] &= \mathrm{e}^{-(T-t) r} \mathbb{E}\left[\left(S_{T}-K\right)^{+} \mid \mathcal{F}_{t}\right] \\
&= S_{t} \Phi\left(\delta_{+}^{T-t}\left(\frac{S_{t}}{K}\right)\right)-K \Phi\left(\delta_{-}^{T-t}\left(\frac{S_{t}}{K}\right)\right)
\end{aligned}
\]

\end{proposition}


\begin{proof}
    See \ref{appendix:a}
\end{proof}




\begin{proposition}
    Price of a double no-touch option with lower and upper barriers $S_{min}$, $S_{max}$:

\[
V(S,t) = 2\pi \left(\dfrac{S}{S_{min}}\right)^{\alpha} e^{\beta(T-t)} \sum_{n=1}^{\infty} n \left[\dfrac{(1 - (-1)^{m}e^{-\alpha H})}{\alpha^2 H^2 + n^2 \pi^2} \right] \exp \left(-\frac{1}{2}\sigma^2 \dfrac{n^2 \pi^2}{H^2} t \right) \sin \left(\dfrac{n\pi}{H} \ln \frac{S}{S_{min}} \right),
\]

with $H=\ln(S_{max}/S_{min})$, $\alpha = -\Tilde{r}/\sigma^2$ and $\beta = -\Tilde{r}/ 2\sigma^2 - r$, $\Tilde{r} = r - \frac{1}{2}\sigma^2$.
\end{proposition}

\begin{proof}
    See \ref{appendix:b}
\end{proof}


