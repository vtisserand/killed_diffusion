\section{Monte Carlo methods}

Our objective is to estimate $\mathbb{E}\left[\mathbbm{1}_{T < \tau} f(X_T)\right]$, as seen before.
Let us introduce a more general diffusion than the Black-Scholes model, the Itô diffusion, as used in the article of Emmanuel Gobet:
% (ajouter la source)!!!!!

Let $\left(X_{t}\right)_{t \geq 0}$ be the Itô diffusion taking its values in $\mathbb{R}^{d}$ defined by

$$
\begin{aligned}
X_{t}=x+\int_{0}^{t} B\left(X_{s}\right) d s+\int_{0}^{t} \sigma\left(X_{s}\right) d W_{s}
\end{aligned}
$$

where $\left(W_{t}\right)_{t \geq 0}$ is a Brownian motion in $\mathbb{R}^{d^{\prime}}$. Let $\tau:=\inf \left\{t>0: X_{t} \notin D\right\}$ be its first exit time from the open set $D \subset \mathbb{R}^{d}$. 

We need to make assumptions on the regularity of the coefficients $B$ and $\sigma$ in order to ensure the existence and uniqueness of the solution to the SDE. We also need to make assumptions on the boundary of the open set $D$ in order to ensure the existence of the first exit time $\tau$.

In the case of our barrier option, we can take $D = (-\infty, b)$, where $b$ is the barrier level. We can also take $D = (a, +\infty)$, or $D = (a, b)$, where $a$ and $b$ are the lower and upper barriers, respectively, in the case of a double no-touch barrier option.

% Notes personnelles :
% Introduction : temps d'arrêt de sortie -> comment l'estimer ? discret / continue ... (OK)
% Jongler avec les hypothèses : bien les retranscrire dans le cas qui nous intéresse où D = (-inf, b) (OK)
% Présenter les hypothèses H1, H2, H3, H4, et H5-2 (TODO)

Using the theory of Monte Carlo numerical methods for the Stochastic Differential Equation (SDE) with a first exit time, we can estimate the price of a barrier option.

We can use the Euler scheme, which is a simple and efficient method to simulate the diffusion process.

The question remains: shall we use a discrete or continuous Euler scheme? The continuous Euler scheme is more accurate, but it is more difficult to simulate the first exit time from the open set $D$.
Indeed between two time steps the diffusion can still unfortunately exit the domain.

\subsection{Methods}

\begin{enumerate}
    \item Discrete Euler scheme


To evaluate the expectation of the functional of the diffusion, the simplest way to approximate the process is to use its discrete Euler scheme $\left(\tilde{X}_{t_{i}}\right)_{0 \leq i \leq N}$ with discretization step $T / N$, defined if $t_{i}=i T / N$ is the $i$-th discretization time by

% ajouter l'accolade
$$ 
\begin{aligned}
\tilde{X}_{0} & =x \\
\tilde{X}_{t_{i+1}} & =\tilde{X}_{t_{i}}+B\left(\tilde{X}_{t_{i}}\right) T / N+\sigma\left(\tilde{X}_{t_{i}}\right)\left(W_{t_{i+1}}-W_{t_{i}}\right)
\end{aligned}
$$

The simulation involves simulation $N$ independent Gaussian variables for the increments $\left(W_{t_{i+1}}-W_{t_{i}}\right)_{0 \leq i \leq N-1}$.

    \item Continuous Euler scheme

Introducing a more complex scheme, our goal is to refine the discrete Euler scheme by considering a modified stopping time $\tilde{\tau}_{c}$, which is the first exit time from the open set $D$ for the continuous Euler scheme.

$\left(\tilde{X}_t\right)_{t_{i} \leq t \leq t_{i+1}}$ shall be defined as:

$$
\begin{aligned}
\tilde{X}_{t} & = \tilde{X}_{t_{i}}+B\left(\tilde{X}_{t_{i}}\right)\left(t-t_{i}\right)+\sigma\left(\tilde{X}_{t_{i}}\right)\left(W_{t}-W_{t_{i}}\right),
\text{ for } t_{i} \leq t < t_{i+1},
\footnote{
    We must rigorously define a new SDE dynamic for the diffusion $\left(\tilde{X}_t\right)$
    with the function $\varphi(t)$ introduced in the article:
    $\tilde{X}_{t}=x+\int_{0}^{t} B\left(\tilde{X}_{\varphi(s)}\right) d s+\int_{0}^{t} \sigma\left(\tilde{X}_{\varphi(s)}\right) d W_{s}$, where $\varphi(t):=\sup \left\{t_{i}: t_{i} \leq t\right\}$. Let $\tilde{\tau}_{c}:=\inf \left\{t: \tilde{X}_{t} \notin D\right\}$ be its first exit time from $D$.
    We now approximate $\mathbb{E}_{x}\left[\mathbbm{1}_{T<\tau} f\left(X_{T}\right)\right]$ by $\mathbbm{E}_{x}\left[\mathbbm{1}_{T<\tilde{\tau}_{c}} f\left(\tilde{X}_{T}\right)\right]$.
}
\end{aligned}
$$

% footnote: add a footnote explaining that we must define a new SDE dynamic with the phi introduced as the article and give the definition of tau_c
% $$
% \begin{equation*}
% \tilde{X}_{t}=x+\int_{0}^{t} B\left(\tilde{X}_{\varphi(s)}\right) d s+\int_{0}^{t} \sigma\left(\tilde{X}_{\varphi(s)}\right) d W_{s} \tag{4}
% \end{equation*}
% $$
% where $\varphi(t):=\sup \left\{t_{i}: t_{i} \leq t\right\}$. Let $\tilde{\tau}_{c}:=\inf \left\{t: \tilde{X}_{t} \notin D\right\}$ be its first exit time from $D$. We are also interested in studying the approximation of $\mathbb{E}_{x}\left[\mathbbm{1}_{T<\tau} f\left(X_{T}\right)\right]$ by $\mathbb{E}_{x}\left[\mathbbm{1}_{T<\tilde{\tau}_{c}} f\left(\tilde{X}_{T}\right)\right]$.



Between two time steps, the diffusion can still exit the domain. From a simulation standpoint, Gobet suggests drawing $N$ additional independent Bernoulli variables to simulate if $\left(\tilde{X}_{t}\right)_{0 \leq t \leq T}$ has left $D$ between two discretization times or not. Each parameter involved for the simulation of the Bernoulli variables is related to the quantity

$$
\mathbb{P}\left(\forall t \in\left[t_{i}, t_{i+1}\right] \tilde{X}_{t} \in D \vert \tilde{X}_{t_{i}}=z_{1}, \tilde{X}_{t_{i+1}}=z_{2}\right):=p\left(z_{1}, z_{2}, T / N\right)
$$

In the 1-dimensional case, $p\left(z_{1}, z_{2}, \Delta\right)$ is the cumulative of the infimum and supremum of a linear Brownian bridge and has a simple expression\footnote{see \cite{revuz2013continuous}, p. 105, for instance}:

1. if $D=(-\infty, b)$, we have $p\left(z_{1}, z_{2}, \Delta\right)=\mathbbm{1}_{b>z_{1}} \mathbbm{1}_{b>z_{2}}\left(1-\exp \left(-2 \frac{\left(b-z_{1}\right)\left(b-z_{2}\right)}{\sigma^{2}\left(z_{1}\right) \Delta}\right)\right)$;
2. if $D=(a,+\infty)$, we have $p\left(z_{1}, z_{2}, \Delta\right)=\mathbbm{1}_{z_{1}>a} \mathbbm{1}_{z_{2}>a}\left(1-\exp \left(-2 \frac{\left(a-z_{1}\right)\left(a-z_{2}\right)}{\sigma^{2}\left(z_{1}\right) \Delta}\right)\right)$;
3. if $D=(a, b)$, we have

$$
\begin{aligned}
p\left(z_{1}, z_{2}, \Delta\right)= & \mathbbm{1}_{b>z_{1}>a} \mathbbm{1}_{b>z_{2}>a}\left(1-\sum_{k=-\infty}^{+\infty}\left[\exp \left(-2 \frac{k(b-a)\left(k(b-a)+z_{2}-z_{1}\right)}{\sigma^{2}\left(z_{1}\right) \Delta}\right)\right.\right. \\
& \left.\left.-\exp \left(-2 \frac{\left(k(b-a)+z_{1}-b\right)\left(k(b-a)+z_{2}-b\right)}{\sigma^{2}\left(z_{1}\right) \Delta}\right)\right]\right) .
\end{aligned}
$$

\end{enumerate}

\newpage

\subsection{Theoretical results}

Let us introduce the following (weak) convergence errors for the two respective schemes:

\begin{itemize}
    \item Continuous Euler Scheme

$$
\begin{aligned}
\mathcal{E}_{\mathrm{c}}(f, T, x, N):=\mathbb{E}_{x}\left[\mathbbm{1}_{T<\tilde{\tau}_{c}} f\left(\tilde{X}_{T}\right)\right]-\mathbb{E}_{x}\left[\mathbbm{1}_{T<\tau} f\left(X_{T}\right)\right]
\end{aligned}
$$

    \item Discrete Euler Scheme

$$
\begin{aligned}
\mathcal{E}_{\mathrm{d}}(f, T, x, N):=\mathbb{E}_{x}\left[\mathbbm{1}_{T<\tilde{\tau}_{d}} f\left(\tilde{X}_{T}\right)\right]-\mathbb{E}_{x}\left[\mathbbm{1}_{T<\tau} f\left(X_{T}\right)\right]
\end{aligned}
$$
\end{itemize}

% as a function of $N$, the number of discretization steps. We first state an easy result, which shows that both errors tend to 0 when $N$ goes to infinity under mild assumptions.


We are interested in evaluating the errors $\mathcal{E}_{\mathrm{c}}(f, T, x, N)$ and $\mathcal{E}_{\mathrm{d}}(f, T, x, N)$ as a function of $N$, the number of discretization steps.
We first state an easy result, which shows that both errors tend to 0 when $N$ goes to infinity under mild assumptions.
We then provide a more precise result, which gives the rate of convergence of the errors.

\subsubsection{Limit behavior of the weak convergence}

% réecrire sous forme de proposition
Assume that $B$ and $\sigma$ are globally Lipschitz functions, that $D$ is defined by $D=\left\{x \in \mathbb{R}^{d}: F(x)>0\right\}, \partial D=\left\{x \in \mathbb{R}^{d}: F(x)=0\right\}$ for some globally Lipschitz function $F$. Provided that the condition (C) below is satisfied

$$
\mathbf{( C )}: \mathbb{P}_{x}\left(\exists t \in[0, T] \quad X_{t} \notin D ; \forall t \in[0, T] \quad X_{t} \in \bar{D}\right)=0
$$

for all function $f \in C_{b}^{0}(\bar{D}, \mathbb{R})$, we have

$$
\lim _{N \rightarrow+\infty} \mathcal{E}_{\mathrm{c}}(f, T, x, N)=\lim _{N \rightarrow+\infty} \mathcal{E}_{\mathrm{d}}(f, T, x, N)=0 .
$$


\subsubsection{Order of convergence of the weak convergence error}

Under regularity assumptions on $B, \sigma, D$ and an uniform ellipticity condition, one has

\begin{itemize}
    \item for the continuous Euler scheme:

$$
\mathcal{E}_{\mathrm{c}}(f, T, x, N)=C N^{-1}+o\left(N^{-1}\right)
$$

provided that $f$ is a measurable function with support strictly included in $D$ (Theorem 2.1, see below). The support condition can be weakened if $f$ is smooth enough (Theorem 2.2, see below).

    \item for the discrete Euler scheme:

$$
\mathcal{E}_{\mathrm{d}}(f, T, x, N)=O\left(N^{-1 / 2}\right)
$$
\end{itemize}

\subsubsection{Theorems}

We make some reasonable assumptions on the regularity of the coefficients of the diffusion process, the smoothness of the survival domain, and the payoff function.

These allow us to bound the error functions, at the rate $N^{-1}$ for the continuous scheme and the rate $N^{-1/2}$ for the discrete scheme.

\subsection{Proof sketches}

Introducing $v(t,x):=\mathbb{E}_{x}\left[\mathbbm{1}_{T-t<\tau} f(X_{T-t})\right]$, we break the error into two parts to make appear the process and stopping time from the scheme: $\Tilde{X}$ and $\tilde{\tau}$. The difference between the continuous and discrete errors essentially stems from the fact that Itô's formula is applicable on $v$ in the former scheme but not the latter as discontinuities of $v$ on the boundary of the domain $D$ would lead to jumping spatial derivatives for.\\

For instance: 

\begin{align*}
\mathcal{E}_{\mathrm{c}}(f, T, x, N)= & \mathbb{E}_{x}\left[\mathbbm{1}_{T<\tilde{\tau}_{c}} f\left(\tilde{X}_{T}\right)\right]-\mathbb{E}_{x}\left[v\left((T-T / N) \wedge \tilde{\tau}_{c}, \tilde{X}_{(T-T / N) \wedge \tilde{\tau}_{c}}\right)\right] \\
& +\mathbb{E}_{x}\left[v\left((T-T / N) \wedge \tilde{\tau}_{c}, \tilde{X}_{(T-T / N) \wedge \tilde{\tau}_{c}}\right)\right]-\mathbb{E}_{x}\left[v\left(0, \tilde{X}_{0}\right)\right] \\
:= & C_{1}(N)+C_{2}(N)
\end{align*}

and $C_1$ will yield a contribution of the order $O(N^{-3/2})$ while $C_2$ will be broken down between two other error terms, one having a dominating contribution of the order $O(N^{-1})$ -- this involves applying Itô's formula and the Bernstein inequality for martingales.

% Citer l'ensemble des lemmes utilisés dans la section 3. Donner les grands résultats uniquement, et ce, à chaque fois qu'il utilise un théorème du cours. Bien mettre en évidence l'aspect discontinue de la fonction v introduite par Gobet, sur laquelle on ne peut pas faire un Itô classique car discontinuité sur le bord du domaine D... (les dérivées spatiales de v présentent un jump) 